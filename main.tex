\documentclass[APA,Times1COL]{WileyNJDv5}

\usepackage{graphicx}
\usepackage{amsmath}
\usepackage{lineno}
\usepackage{xr-hyper}
% hyperref is already loaded by the document class
% \usepackage[colorlinks=true]{hyperref}



\articletype{Article Type}%

\received{Date Month Year}
\revised{Date Month Year}
\accepted{Date Month Year}
\journal{Journal}
\volume{00}
\copyyear{2023}
\startpage{1}

\raggedbottom

\begin{document}

\linenumbers

\title{Untamed: Unconstrained Tensor Decomposition and Graph Node Embedding for Cortical Parcellation}

\author[1]{Yijun Liu}

\author[2,3]{Jian Li}

\author[4,5]{Jessica L. Wisnowski}

\author[1]{Richard M. Leahy}

\authormark{Liu \textsc{et al.}}
\titlemark{Untamed: Unconstrained Tensor Decomposition and Graph Node Embedding for Cortical Parcellation}

\address[1]{\orgdiv{Ming Hsieh Department of Electrical and Computer Engineering}, \orgname{University of Southern California}, \orgaddress{\state{CA}, \country{United States}}}

\address[2]{\orgdiv{Athinoula A. Martinos Center for Biomedical Imaging}, \orgname{Massachusetts General Hospital and Harvard Medical School}, \orgaddress{\state{Charlestown}, \country{United States}}}

\address[3]{\orgdiv{Center for Neurotechnology and Neurorecovery, Department of Neurology}, \orgname{Massachusetts General Hospital and Harvard Medical School}, \orgaddress{\state{MA}, \country{United States}}}

\address[4]{\orgdiv{Radiology and Pediatrics, Division of Neonatology}, \orgname{Children’s Hospital Los Angeles}, \orgaddress{\state{CA}, \country{United States}}}

\address[5]{\orgdiv{Keck School of Medicine}, \orgname{University of Southern California}, \orgaddress{\state{CA}, \country{United States}}}

\corres{Corresponding author: Richard M. Leahy \email{leahy@sipi.usc.edu}}

% \presentaddress{This is sample for present address text this is sample for present address text.}

%\fundingInfo{Text}
%\JELinfo{ejlje}

\abstract[Abstract]{Cortical parcellation is fundamental to neuroscience, enabling the division of cerebral cortex into distinct, non-overlapping regions to support interpretation and comparison of complex neuroimaging data. Although extensive literature has investigated cortical parcellation and its connection to functional brain networks, the optimal spatial features for deriving parcellations from resting-state fMRI (rsfMRI) remain unclear. Traditional methods such as Independent Component Analysis (ICA) have been widely used to identify large-scale functional networks, while other approaches define disjoint cortical parcellations. However, bridging these perspectives through effective feature extraction remains an open challenge. To address this, we introduce \textit{Untamed}, a novel framework that integrates unconstrained tensor decomposition using NASCAR to identify functional networks, with state-of-the-art graph node embedding to generate cortical parcellations. Our method produces near-homogeneous, spatially coherent regions aligned with large-scale functional networks, while avoiding strong assumptions like statistical independence required in ICA. Across multiple datasets, Untamed consistently demonstrates improved or comparable performance in functional connectivity homogeneity and task contrast alignment compared to existing atlases. The pipeline is fully automated, allowing for rapid adaptation to new datasets and the generation of custom parcellations. The atlases derived from the Genomics Superstruct Project (GSP) dataset, along with the code for generating customizable parcel numbers, are publicly available at \href{https://untamed-atlas.github.io}{https://untamed-atlas.github.io}.}

\keywords{Cortical parcellation, Resting-state fMRI, Temporal synchronization, Tensor decomposition, Graph representation learning}

\jnlcitation{\cname{%
\author{Yijun Liu},
\author{Lauritzen P},
\author{Erath C}, and
\author{Mittal R}}.
\ctitle{On simplifying ‘incremental remap’-based transport schemes.} \cjournal{\it J Comput Phys.} \cvol{2021;00(00):1--18}.}


\maketitle

\renewcommand\thefootnote{}
% \footnotetext{\textbf{Abbreviations:} ANA, anti-nuclear antibodies; APC, antigen-presenting cells; IRF, interferon regulatory factor.}

% \renewcommand\thefootnote{\fnsymbol{footnote}}
% \setcounter{footnote}{1}


\section{Introduction}

Elucidating the macrostructure of the human brain remains a cornerstone
in neuroscience. From the early work of Brodmann though Talairach
through the most recent efforts \cite{Brodmann1909, TzourioMazoyer2002, Schaefer2018, Eickhoff2018, Han2020}, a key aim has been identifying the parcels -- the relatively homogenous mesoscale computational units that together give rise to the large-scale neural networks of the human brain.

The most common approach to elucidating the intrinsic macrostructure
involves resting-state functional magnetic resonance imaging (rsfMRI).
By recording blood-oxygen-level-dependent (BOLD) signals without task stimuli, rsfMRI proved instrumental in deciphering the human brain's intrinsic functional brain networks \cite{allen2014, Beckmann2005, biswal1995}. From these, investigators have derived both group-level \cite{Yeo2011, Craddock2012, Shen2013, Gordon2016, Schaefer2018, Yan2023}, and individual \cite{Chong2017, Kong2019} parcellations of the human brain. However, until recently, resting-state networks have been constrained by statistical independence or orthogonality, which are inherent to independent component analysis (ICA) or principle component
analysis (PCA) approaches. Yet, these constraints may not be consistent
with the intrinsic functional brain networks \cite{Harris2013, Rockland2019}.

We have developed a new way for identifying overlapping and
non-orthogonal functional networks using a Nadam-Accelerated SCAlabe and
Robust (NASCAR) tensor decomposition method \cite{Li2021, Li2023}. The
aim of this study is to determine whether cortical parcellations,
derived from NASCAR networks, would be more robust when compared to
prior parcellations. Briefly, NASCAR computes a three-way tensor
decomposition of temporally aligned rsfMRI data across subjects \cite{Joshi2018, akrami2019a} without imposing implausible
constraints, such as statistical independence or orthogonality \cite{Smith2014}. It facilitates the identification of functional networks
that may more authentically represent brain activity patterns by
avoiding the need to impose spatial and/or temporal independence between
networks. The networks identified by NASCAR have been shown to be, to a
large degree, consistent with those found using ICA but with subtle
differences that are consistent with other findings in the literature,
such as subsystems of the default-mode network found using seed-based
clustering and Bayesian methods \cite{andrewsHanna2010, AndrewsHanna2014, Buckner2019, Harrison2020}. We also showed that the networks found using NASCAR could
be used to better classify subjects with neurological disorders \cite{Li2023}.

Based on the brain networks identified by NASCAR, we construct a graph
and apply a graph node embedding method, NetMF \cite{Qiu2018}, to
generate cortical parcellations. This integration of tensor
decomposition and graph node embedding enables the production of
parcellations that are functionally homogeneous, computationally
efficient, and adaptable to a wide range of datasets. Unlike
parcellation methods that rely on manual input (e.g., \cite{glasser2016, Joshi2022}), our pipeline is fully automated, enabling researchers to seamlessly apply the framework to new datasets with a
customizable number of parcels. This automation facilitates broader
exploration of neuroscience questions, such as examining the influence
of functional network spatial maps on the resulting parcels and
determining the optimal number of parcels, as demonstrated in our
experiments.

Our approach, which we term \emph{Untamed}
(\textbf{\underline{Un}}constrained \textbf{\underline{t}}ensor
decomposition-based \textbf{\underline{a}}ctivation
\textbf{\underline{m}}aps and emb\textbf{\underline{ed}}ding), delivers
performance that is comparable to or exceeds widely used parcellations
(e.g., \cite{Schaefer2018}) across two critical metrics:
resting-state functional connectivity homogeneity and task contrast
alignment. By addressing methodological challenges in determining
spatial features and linking cortical parcellation with functional
networks, Untamed provides a robust framework for advancing research in
neuroscientific studies. Both the parcellations derived from the
Genomics Superstruct Project \cite{Holmes2015} dataset and the code
to generate customized atlases are available publicly at
\url{https://untamed-atlas.github.io}.

\hypertarget{material-and-methods}{%
\section{Material and methods}\label{material-and-methods}}

\hypertarget{overview}{%
\subsection{Overview}\label{overview}}

We derived cortical parcellations using rsfMRI data from 1428 subjects from the Genomics Superstruct Project (GSP) dataset \cite{Holmes2015}. The overall pipeline is depicted in Fig.~\ref{fig:untamed_framework}. The procedure
included a temporal alignment using BrainSync \cite{akrami2019a, Joshi2018}, a 3-way tensor decomposition using NASCAR \cite{Li2021, Li2023}, graph construction from spatial maps followed by graph
node embedding using NetMF \cite{Qiu2018}, and finally \(k\)-means
clustering to obtain the parcellations. We evaluated the parcellation
performance on the Yale rsfMRI \cite{Lee2022}, Human Connectome
Project (HCP) \cite{VanEssen2012, Glasser2013}, and the
multi-domain task battery (MDTB) \cite{King2019, Zhi2022} datasets, with varied acquisition protocols and data modalities.

\hypertarget{datasets}{%
\subsection{Datasets}\label{datasets}}

\hypertarget{the-genomics-superstruct-project-gsp-dataset}{%
\subsubsection{The Genomics Superstruct Project (GSP)
dataset}\label{the-genomics-superstruct-project-gsp-dataset}}

We utilized the Genomics Superstruct Project (GSP) \cite{Holmes2015} dataset for atlas generation. The dataset contains 3T resting-state fMRI
(rsfMRI) scans from 1570 subjects (665 males, 905 females, age between
22 and 35). Each subject underwent either one or two rsfMRI scans. Of
the 1570 participants, 1139 completed two scans, while 431 completed
only a single scan. The rsfMRI data were acquired with \(TR = 3\)s, and
a 3 mm isotropic spatial resolution. Each session lasted 6 min and 12 s,
yielding 124 time points. The structural data associated with each
subject consisted of a single 1.2 mm isotropic scan.

Data preprocesssing was conducted using the publicly available pipeline
from the CBIG repository
(\url{https://github.com/ThomasYeoLab/Standalone_CBIG_fMRI_Preproc2016}),
configured to closely align with methods described in \cite{Schaefer2018, Li2019, Yan2023}. Each subject's T1-weighted
image was registered to the standard MNI 2mm space. Key preprocessing
steps of fMRI data included slice time correction, motion correction,
censoring of outlier volumes, signal regression of white matter,
ventricular signals, whole brain signal, and bandpass filtering (0.009
Hz -- 0.08 Hz). Alignment between fMRI and structural images was
performed using boundary-based registration \cite{Greve2009}.
The preprocessed fMRI were projected to the Freesurfer's fsaverage6
space and smoothed using a Gaussian kernel with a 6 mm full-width
half-maximum (FWHM). Subjects with at least one rsfMRI session that
passed quality control checks in the preprocessing pipeline were
included in subsequent experiments. This resulted in a final sample of
1428 subjects (603 males, 825 females), of which 1034 had two scans and
394 had one scan.


\subsubsection{The Human Connectome Project (HCP)
dataset}

We utilized the minimally preprocessed 3T rsfMRI data from 1000 subjects
(466 males, 534 females, age between 22 and 35) from the Human Connec
tome Project (HCP) \cite{Glasser2013, VanEssen2012} for
the assessment of parcellations. The rsfMRI data were acquired with
\(TR = 0.72\) s, \(TE = 33.1\) ms, and a 2 mm isotropic resolution and
co-registered onto a common atlas in MNI space \cite{Glasser2013}.
We used the scans acquired in the LR phase encoding direction. Each
session ran 15 minutes with 1200 time points. The data were resampled
onto the cortical surface extracted from each subject's T1-weighted MRI
and co-registered to a common surface \cite{Glasser2013}. No
additional spatial smoothing was applied other than the 2 mm FWHM
isotropic Gaussian smoothing in the minimal preprocessing pipeline
\cite{Glasser2013}.

In addition to rsfMRI data, we utilized subject-wise task activation
z-score maps in fs\_LR 32K surface space from the HCP dataset \cite{barch2013} for evaluation. These maps covered seven task domains:
working memory, gambling, motor, language, social, relational, and
emotion. Task contrast maps were derived from 3T task fMRI data acquired
with the same parameters as the rsfMRI data, except for the duration
which varied based on the particular task. We included all available
contrasts and incorporated data from 1,006 subjects with complete task
datasets. Furthermore, group-level task activation z-score maps for the
same tasks were also employed for the analysis of optimal parcel
numbers.

\hypertarget{yale-resting-state-fmri-dataset}{%
\subsubsection{Yale resting-state fMRI
dataset}\label{yale-resting-state-fmri-dataset}}

We used the Yale rsfMRI dataset \cite{Lee2022} as an independent
dataset for assessing RSFC homogeneity \cite{Schaefer2018}. The
dataset comprises 27 subjects (11 males, 16 females, aged between 22 and
31). Each subject underwent two rsfMRI scans, each with \(TR = 1\)s,
\(TE = 30\)ms, and a 2 mm isotropic resolution. The total duration of
each session was 6 minutes 50 seconds, encompassing 410 frames. We
preprocessed the data using fMRIPrep \cite{Esteban2019}, which
sampled the fMRI data onto the fs\_LR 32K surfaces compatible with the
HCP dataset. Four subjects' data were not successfully preprocessed by
the fMRIPrep pipeline due to corrupted files and were consequently
excluded from the evaluation. Following the recommendations from the
curators of the dataset, we discarded the first 10 seconds of each scan
and temporally concatenated the rsfMRI data of the two sessions for each
subject.

\hypertarget{multi-domain-task-battery-mdtb-dataset}{%
\subsubsection{Multi-domain task battery (MDTB)
dataset}\label{multi-domain-task-battery-mdtb-dataset}}

We employed the multi-domain task battery (MDTB) dataset \cite{King2019, Zhi2022} for evaluating the alignment between parcels in
the atlases and areas of activation in the task activation maps. The
MDTB dataset contains task fMRI for 24 healthy subjects (8 males, 16
females, mean age 23.8 years old) conducting 26 tasks, including motor,
language, and social domains. The fMRI data were acquired on a 3T
Siemens Prisma scanner with \(\text{TR} = 1s\), 3 mm slice thickness,
and \(2.5 \times 2.5\ mm^{2}\) in-plane resolution. The contrast maps
were derived using general linear modeling (GLM) based on the task
designs and are available from the database, already resampled to the
fs\_LR 32K space.


\subsection{Tensor-based identification of Brain Networks using NASCAR and BrainSync}\label{sec:identify_brain_network}

% \begin{figure*}
%     \centering
%     \includegraphics[width=6.6in]{figs/Fig_1.jpg}
%     \caption{Overview of the Untamed parcellation framework. NASCAR tensor decomposition yields spatial network maps $\bm{X} \in \mathbb{R}^{|\bm{V}|\times n}$, where $|\bm{V}|$ is the number of cortical vertices and $n$ is the number of functional networks. For each hemisphere, a graph $(\bm{V}_\cdot, \bar{\bm{A}}_\cdot)$ is constructed based on the pairwise correlations of network participation across vertices. Graph node embeddings $\bm{B} \in \mathbb{R}^{|\bm{V}_\cdot|\times d}$ are derived using NetMF,  producing $d$-dimensional representations for each vertex. Finally, clustering in the embedding space yields parcellations $\bm{c}_{|\bm{V}|}$, i.e., assignments of vertices into parcels at the chosen resolution.}
%     \label{fig:untamed_framework}
% \end{figure*}


For each subject, if two sessions of rsfMRI data were available, the time series from the two sessions were concatenated temporally. If only
one session was available, the session was duplicated and concatenated.
The resulting multisubject rsfMRI data were then organized as a
three-way tensor (space \(\times\) time \(\times\) subject). In order to
identify a low-rank model via tensor decomposition, the data needed to
be both temporally and spatially aligned. As described in the
preprocessing pipeline for the GSP dataset, inter-subject spatial
alignment was achieved using standard image registration methods,
mapping each subject's data to the standard fsaverage6 surface space.
However, while spatial alignment ensured that brain regions corresponded
across subjects, the rsfMRI time series remained independent for each
subject. To address this we assume that subjects share a similar
connectivity structure as reflected in the pairwise correlation patterns
between brain regions. We can then apply an orthogonal transformation to
temporally align or synchronize the data across subjects. This temporal alignment is achieved using the BrainSync transform \cite{Joshi2018}. BrainSync computes an orthogonal transformation between fMRI
recordings from a pair of subjects such that the sum of squared errors
between their aligned time series is minimized. As a result, the time
series in homologous brain locations will be highly correlated after the
BrainSync transform is applied (and perfectly so if the two datasets
have identical correlations). A multi-subject extension, as described in
\cite{akrami2019a}, minimizes the squared error from all subjects to
an automatically generated group template. This method was used in this
study to align all subjects for generating the cortical parcellation.

After temporal synchronization, we formed a third-order tensor
\(\mathcal{X} \in \mathbb{R}^{\mathbf{|V|} \times T \times S}\) from the
spatially aligned and temporally synchronized rsfMRI data, where
\(\mathbf{V}\) is the set of cortical vertices with a cardinality
\(\mathbf{|V|} \approx 75K\), \(T = 240\) is the number of time points,
\(S = 1428\) is the number of subjects used to generate the atlas. We
then applied NASCAR to decompose the group rsfMRI data into a set of
brain networks common to all subjects using a Canonical Polyadic model
\cite{Kolda2009, Cichocki2015, Li2021, Li2023}


\begin{equation}
    \mathcal{X} \approx \sum_{i = 1}^{R}{{\lambda_{i}\mathbf{\ }\mathbf{a}}_{i} \circ \mathbf{b}_{i} \circ \mathbf{c}_{i}}
\end{equation}

where each outer product
\(\lambda_{i}\ \mathbf{a}_{i} \circ \mathbf{b}_{i} \circ \mathbf{c}_{i}\)
represents a brain network:
\(\mathbf{a}_{i} \in \mathbb{R}^{\mathbf{|V|}}\) are the spatial network
maps, \(\mathbf{b}_{i} \in \mathbb{R}^{T}\) their (synchronized)
temporal dynamics, and \(\mathbf{c}_{i} \in \mathbb{R}^{S}\) the subject
participation level for the \(i^{\text{th}}\) network; \(\lambda_{i}\)
represents the network magnitude, indicating the relative activity level
with respect to other networks. In contrast to the commonly used blind
source separation techniques such as PCA or ICA, NASCAR imposes neither
orthogonality (as with PCA)~\cite{Smith2014} nor statistical
independence (as with ICA)~\cite{Beckmann2005}. The shared spatial
network maps \(\mathbf{\{ a}_{i}\}\) can be overlapped and correlated
\cite{Li2023}. Based on our previous work that examined stability
and reproducibility across data sets, we chose a rank \(R\)=50~\cite{Li2023}. We performed NASCAR decomposition on the whole GSP dataset
and obtained a set of 50 spatial network maps, which formed a spatial
feature matrix \(\mathbf{X \in}\mathbb{R}^{\mathbf{|V|} \times 50}\mathbf{\ (|V|} \approx 75K)\).
This matrix provided a 50-dimensional feature vector for each vertex.
Specifically, \(\mathbf{X}_{ij}\) represented the participation
level at the \(i\)\textsuperscript{th} vertex in the
\(j\)\textsuperscript{th} network map.


\subsection{Graph construction from NASCAR spatial
maps}
\label{sec:graph_construct_NASCAR}

We computed the Pearson correlation matrix
\(\mathbf{A} = corr(\mathbf{X}) \in \mathbb{R}^{\mathbf{|V|} \times \mathbf{|V|}\ }\)
from the feature matrix \(\mathbf{X}\) as a measure of similarity
between feature vectors of each pair of vertices. We then computed an
adjacency matrix, following \cite{Ng2001}, using the Gaussian kernel
\({\widetilde{\mathbf{A}}}_{i, j} = \exp\left( \frac{\mathbf{A}_{i, j}}{2\sigma^{2}} \right)\), \(\ \sigma = 0.5\). Our goal was to produce contiguous parcels, and as
such, we were not interested in long-range correlations. Therefore, we
generated a graph where connections were restricted using an
\(\text{nb}\)-hop spatial neighborhood constraint defined on the fs\_LR
32K surface mesh. Specifically, we defined \({\overline{\mathbf{A}}}_{i, j} = {\widetilde{\mathbf{A}}}_{i, j}\) for all \((i,j)\) for which \(i\) is within \(\text{nb}\) hops of \(j\) and zero otherwise. In a related approach, \cite{Craddock2012} used
a \(1\)-hop neighborhood, retaining nearest neighbors only. However, we
found that a larger \(\text{nb}\) could produce higher RSFC homogeneity
(see Evaluation section). Consequently, rather than fixing this
parameter, we treated it as a hyperparameter within the algorithm,
optimized using the GSP dataset. Details of the hyperparameter selection
process are provided in Section~\ref{sec:graph-node-embedding-and-clustering}.

Using the above approach, we obtained one sparsely connected (\(m\)-hop)
graph for each hemisphere:
\(\mathbf{G}_{L} = (\mathbf{V}_{L},{\overline{\mathbf{A}}}_{L})\) and
\(\mathbf{G}_{R} = (\mathbf{V}_{R},{\overline{\mathbf{A}}}_{R})\), where
\(\mathbf{V}_{L}\), \(\mathbf{V}_{R}\) were the set of cortical vertices
and \({\overline{\mathbf{A}}}_{L}\),
\({\overline{\mathbf{A}}}_{R}\ \)were the \(\text{nb}\)-hop filtered
adjacency matrices of the graphs \(\mathbf{G}_{L}\), \(\mathbf{G}_{R}\),
for the left and the right hemispheres, respectively.

\subsection{Graph node embedding and
clustering}
\label{sec:graph-node-embedding-and-clustering}

We adopted the method outlined in NetMF \cite{Qiu2018} to embed
vertices in a lower dimensional space with more representative features
that can be used for clustering. This approach \cite{Qiu2018} established an equivalence between the widely-used DeepWalk algorithm
\cite{Perozzi2014} and matrix factorization techniques. Briefly,
DeepWalk traverses vertices in the graph using a random walk. At each
vertex during traversal, DeepWalk considers neighboring vertices as
co-occurrent ``positive'' pairs as well as random node pairs outside the
neighborhood as ``negative samples.'' It then applies a skip-gram model
to train on the collected samples to derive the final embeddings. NetMF
approximates DeepWalk by factorizing the following matrix \cite{Qiu2018}:

\begin{equation}
    \mathbf{M} = \log\left( \max\left\{ \frac{\sum_{i = 1}^{|\mathbf{V}|}{\sum_{j = 1}^{|\mathbf{V}|}{\overline{\mathbf{A}}}_{i, j}}}{\text{bT}}\left. \ \sum_{r = 1}^{T}{{\ (\mathbf{D}}^{- 1}\overline{\mathbf{A}}\mathbf{)}^{r}{\ \mathbf{D}}^{- 1}\mathbf{,}}\ 1\  \right\} \right.\  \right)
\end{equation}

where\(\ \mathbf{D}\) is the degree matrix of the graph;
\(\log(\cdot)\) and
\(\max\mathbf{\{}{\cdot ,\ }1\mathbf{\}\ }\)are element-wise
logarithmic and maximum operations, respectively; \(T\) is the window
length over which nodes considered as co-occurrent positive pairs; and
\(b\) denotes the expected number of unrelated node pairs each positive
pair is contrasted against when learning its embedding. We found that in
practice the method was not sensitive to the choice of these two
hyperparameters, which had little impact on the averaged training subjects' RSFC homogeneity (see evaluation metrics section). We set $b = 1$, the default value in the original NetMF paper. We computed
the $\mathbf{M}$ matrices for left hemisphere from
${\overline{\mathbf{A}}}_{L}$ and for right hemisphere from \({\overline{\mathbf{A}}}_{R}\) separately. We used singular value
decomposition (SVD) to obtain the final embeddings.
Specifically, let \(\mathbf{M} = \mathbf{U}_{d}\mathbf{\Sigma}_{d}\mathbf{V}_{d}^{\top}\),
where \(\mathbf{\Sigma}_{d}\) and \(\mathbf{U}_{d}\) being the \(d\)
largest singular values and their associated left singular vectors. The
final embeddings were computed as
\(\mathbf{B = U}_{d}\sqrt{\mathbf{\Sigma}_{d}}\) for the left
\((\mathbf{B}_{L}\mathbf{)}\) and right hemisphere \(\mathbf{(B}_{R})\)
separately, with \(d\) being the embedding dimension.

To obtain the final cortical parcellation, we applied \(k\)-means
clustering to the embedding vectors formed by the columns of
\(\mathbf{B}_{L}\) and \(\mathbf{B}_{R}\) and varied the number of
classes \(k\) to match the number of desired parcels in the
parcellation. For each value of \(k\), we ran\(\text{\ k}\)-means
clustering with 500 different random initializations with a maximum of
20000 iterations and selected the result that yielded the smallest cost
across trials as the final output.

The graph construction and embedding procedure involved fine tuning the
following parameters: spatial neighborhood constraint \(\text{nb}\), the
window length \(T\), and the embedding dimension \(d\). Specifically,
\(nb\) was varied across the set of $\{1, 5, 10, 15, 20, \ldots, 60\}$, \(T\)
across \{1, 7, 11, 15\}, and \(d\) across \{128, 256\}. The finetuning
process aimed to maximize the weighted average RSFC homogeneity.
Hyperparameter combinations were selected based on the mean weighted
average RSFC homogeneity across GSP subjects, which served as a summary
statistic to assess performance on the training set.

The following parameter combinations were selected based on the fine
tuning results:

\begin{itemize}
\item
  Embedding dimension = 128
\item
  Neighborhood size (\(nb\)) and window length (\(T\)):
  \begin{itemize}
  \item
    \(nb = 55,\ T = 7\) for parcel number \(\in (0,\ 200\rbrack\)
  \item
    \(nb = 40,\ T = 15\) for parcel number \(\in (200,\ 300\rbrack\)
  \item
    \(nb = 35,\ T = 15\) for parcel number \(\in (301,\ 400\rbrack\)
  \end{itemize}
\end{itemize}

\hypertarget{evaluation-metrics}{%
\subsection{Evaluation metrics}\label{evaluation-metrics}}

\hypertarget{resting-state-functional-connectivity-rsfc-homogeneity}{%
\subsubsection{Resting-state functional connectivity (RSFC)
homogeneity}\label{resting-state-functional-connectivity-rsfc-homogeneity}}

Each subject's RSFC was computed as the Pearson correlation of rsfMRI
time series between all pairs of cortical vertices. A Fisher z-transform
was subsequently performed to obtain z scores from the correlation
values. Similar to the evaluation procedure in \cite{Joshi2022, Schaefer2018}, a parcel-wise homogeneity score \(\rho_{i}\) was
computed as the averaged RSFC values within the
\(i\)\textsuperscript{th} parcel. To obtain a global homogeneity measure
for a single subject, a weighted average \(\rho\) of each parcel's
homogeneity scores was then computed by accounting for different cluster
sizes:

\begin{equation}
    \rho = \sum_{i = 1}^{N}\rho_{i}\frac{|\mathbf{V}^{i}|}{|\mathbf{V}|}
\end{equation}

where \(|\mathbf{V}^{i}|\) is the number of vertices in parcel ~\(i\),
and \(N\) is the total number of parcels;~\(|\mathbf{V}|\) is the total
number of cortical vertices\(.\) For conciseness, we refer to this RSFC
weighted average homogeneity as ``homogeneity'' hereafter. The
homogeneity was computed for each test subject separately. To
quantitatively compare the homogeneity of Untamed to that of other
atlases, we performed a paired-sample t-test. Effect size was reported
using Cohen's \(d\) measure.

\hypertarget{alignment-with-task-contrasts}{%
\subsubsection{Alignment with task
contrasts}\label{alignment-with-task-contrasts}}

The degree to which the regions delineated in a particular parcellation
reflect functional specialization was assessed by computing the variance
\(\sigma_{i}^{2}\) of the task contrast within each parcel. The better
the parcellation delineates regions of functional homogeneity, the lower
the variance of task contrasts within each parcel. Similar to the
procedure in \cite{Schaefer2018, Joshi2022, Yan2023}, the variance metric was first computed for each parcel, and then a weighted average was computed, accounting for parcel size differences,
as:

\begin{equation}
    \sigma^{2} = \sum_{i = 1}^{N}\sigma_{i}^{2}\frac{|\mathbf{V}^{i}|}{|\mathbf{V}|}
\end{equation}

We refer to this weighted average task contrast variance as ``task
variance'' hereafter. The variance was computed separately for each task
contrast and for each subject. These variances were first averaged
across contrasts within each task and then averaged across tasks,
consistent with the approach in \cite{Yan2023}. Statistical
significance was assessed using paired-sample t-test to compare the task
variance between different parcellations.

\hypertarget{comparison-with-existing-parcellations}{%
\subsection{Comparison with existing
parcellations}\label{comparison-with-existing-parcellations}}

We compared our Untamed parcellations to an extensive set of 14 atlases
listed in Table~1. Because comparison of the evaluation metrics defined
above was only meaningful when comparing parcellations of the same
parcellation resolution, in each case, we matched the number of parcels
found using Untamed to the number in the left and right hemispheres for
each baseline comparison.

The evaluations were conducted in the HCP fs\_LR 32K surface space.
Untamed, originally constructed in the fsaverage 6 surface space, was
projected onto the HCP fs\_LR 32K space. For other atlases that were
originally constructed in spaces different from HCP fs\_LR 32K (e.g.,
MNI152, fsaverage), we utilized versions of the atlases that had been
resampled onto the HCP fs\_LR 32K surfaces. These resampled versions
were provided either by the original authors or by third parties, as
detailed in Table~\ref{tab:atlas_baselines}. For the original two atlases proposed by \cite{Yeo2011} that are spatially distributed across hemispheres with a
relatively small number (7 and 17) of networks, we used the version
provided in their GitHub repository (Table~\ref{tab:atlas_baselines}), where the distributed spatial networks were decomposed into local contiguous parcels (51
parcels for the 7 networks and 114 parcels for the 17 networks).

\begin{table*}[!t]%
\centering %
\caption{This is sample table caption.\label{tab2}}%
\begin{tabular*}{\textwidth}{@{\extracolsep\fill}lllll@{\extracolsep\fill}}
\toprule
\textbf{Col1 head} & \textbf{Col2 head}  & \textbf{Col3 head}  & \textbf{Col4 head}  & \textbf{Col5 head} \\
\midrule
col1 text & col2 text  & col3 text  & col4 text  & col5 text\tnote{$^\dagger$}   \\
col1 text & col2 text  & col3 text  & col4 text  & col5 text   \\
col1 text & col2 text  & col3 text  & col4 text  & col5 text\tnote{$^\ddagger$}   \\
\bottomrule
\end{tabular*}
\begin{tablenotes}
\item[$^\dagger$] Example for a first table footnote.
\item[$^\ddagger$] Example for a second table footnote.
\item {\it Source}: Example for table source text.
\end{tablenotes}
\end{table*}



% % \begin{longtable}{@{}c c c c p{5.5cm}@{}}
% \begin{longtable}{@{}c c c c >{\centering\arraybackslash\footnotesize}p{5.5cm}@{}}
% \caption{List of atlases included in the comparison. “Functional”: built from fMRI; “Hybrid”: built from both fMRI and anatomical information.}
% \label{tab:atlas_baselines} \\

% \toprule
% \textbf{Name} & \textbf{Type} & \textbf{\# Clusters} & \textbf{Reference} & \textbf{Atlas Source} \\
% \midrule
% \endfirsthead

% \toprule
% \textbf{Name} & \textbf{Type} & \textbf{\# Clusters} & \textbf{Reference} & \textbf{Atlas Source} \\
% \midrule
% \endhead

% \midrule
% \multicolumn{5}{r}{Continued on next page} \\
% \midrule
% \endfoot

% \bottomrule
% \endlastfoot

% Yeo-51  & Functional & 51 (L: 26, R: 25)   & \cite{Yeo2011} & \url{https://github.com/ThomasYeoLab/CBIG/tree/master/stable_projects/brain_parcellation} \\ \hline
% Yeo-114 & Functional & 114 (L: 57, R: 57)  & \cite{Yeo2011} & \url{https://github.com/ThomasYeoLab/CBIG/tree/master/stable_projects/brain_parcellation} \\ \hline
% Power   & Functional & 130 (L: 65, R: 65)  & \cite{Arslan2018} & \url{https://biomedia.doc.ic.ac.uk/brain-parcellation-survey/} \\ \hline
% USCBrain& Hybrid     & 130 (L: 65, R: 65)  & \cite{Joshi2022} & \url{https://github.com/ajoshiusc/bfp/blob/main/supp_data/USCBrain_grayordinate_labels_clean.mat} \\ \hline
% Shen    & Functional & 200 (L: 102, R: 98) & \cite{Shen2013} & \url{https://biomedia.doc.ic.ac.uk/brain-parcellation-survey/} \\ \hline
% Schaefer & Functional &
% \makecell[t]{100 (L: 50, R: 50)\\200 (L: 100, R: 100)\\300 (L: 150, R: 150)\\400 (L: 200, R: 200)}
% & \cite{Schaefer2018} & \url{https://github.com/ThomasYeoLab/CBIG/tree/master/stable_projects/brain_parcellation} \\ \hline
% Yan & Functional &
% \makecell[t]{100 (L: 50, R: 50)\\200 (L: 100, R: 100)\\300 (L: 150, R: 150)\\400 (L: 200, R: 200)}
% & \cite{align}  & \url{https://github.com/ThomasYeoLab/CBIG/tree/master/stable_projects/brain_parcellation} \\ \hline
% Glasser & Hybrid & 360 (L: 180, R: 180) & \cite{glasser2016} & \url{https://balsa.wustl.edu/study/show/RVVG} \\

% \end{longtable}



\subsection{Ablation study of input spatial maps: NASCAR versus
ICA}
\label{ablation-study-of-input-spatial-maps-nascar-versus-ica}

In the Untamed framework, the graph was constructed using correlations
between features representing the participation of each vertex in the
NASCAR spatial network maps. However, the input spatial maps are not
limited to NASCAR and can be derived from any network decomposition
method. To investigate whether the non-orthogonal spatial maps generated
by NASCAR lead to better performance compared to spatial maps from other
methods, all steps in the pipeline were kept consistent, with the only
variation being the input spatial maps.

For this study, ICA spatial maps provided by HCP were used as an
alternative to NASCAR maps. Both the ICA and NASCAR maps consisted of 50
networks, with no additional selection or preprocessing applied. The
pipeline's hyperparameters---including the number of spatial
neighborhood constraints, the window length in the NetMF algorithm, and
the embedding dimension---were optimized specifically for the ICA maps
and explored within the same range as those used for NASCAR. RSFC
homogeneity was evaluated on the HCP rsfMRI dataset consisting of 1000
test subjects using the same procedure described in Section ~\ref{sec:metric_rsfc_homo}. To
statistically compare the homogeneity values derived from the NASCAR and
ICA spatial maps, a paired-sample t-test was conducted. This analysis
was designed to evaluate how the choice of input spatial maps influences
the performance of the Untamed framework.

We note that the NASCAR spatial maps that Untamed atlases generated from
were derived from the independent GSP dataset in the fsaverage 6 surface
space and subsequently projected onto the HCP fs\_LR 32K surface space.
In contrast, the ICA maps from HCP inherently benefit from
dataset-specific fine-tuning and remain within the same surface space,
avoiding potential information loss associated with the projection
process.

\hypertarget{ablation-study-of-graph-embedding-methods}{%
\subsection{Ablation study of graph embedding
methods}\label{ablation-study-of-graph-embedding-methods}}

We also compared Untamed with the spectral clustering described in \cite{Ng2001}, which applies a \(k\)-means clustering to the most
significant eigenvectors of the normalized graph Laplacian. All steps in
the learning procedure were identical except for the graph embedding,
where we used NetMF. We applied both methods to the same graph
constructed from NASCAR spatial maps as described in Section~\ref{sec:graph_construct_NASCAR}.
Parcellations generated from NetMF and graph Laplacian were subject to
the same neighborhood constraints, embedding dimension, and k-means
clustering hyperparameters with a range of cluster numbers. We evaluated
weighted average homogeneity on the 1000 HCP rsfMRI dataset test
subjects. A paired-sample t-test test was used to compare the
homogeneity values obtained from the Laplacian eigenvectors used in
spectral clustering and the NetMF method that was utilized in Untamed.


\subsection{Ablation study of graph construction
methods}\label{sec:ablation-study-of-graph-construction-methods}

In Untamed, we constructed the graph using the correlation between
features representing the participation of each vertex in each of the
NASCAR spatial network maps. Here, we conducted a comparison to the
graph constructed from the widely used correlation map of RSFC, i.e. the
correlation of the correlation between rsfMRI time series. Other steps
were identical except for the construction of the graph. Parcellations
generated from NASCAR spatial maps and RSFC correlation maps were
subject to the same neighborhood constraints and embedding dimension. We
evaluated weighted average homogeneity on the averaged RSFC from the 500
training subjects in the HCP dataset. A paired-sample t-test was
employed to statistically compare homogeneity values obtained from
NASCAR spatial maps and the correlation of RSFC.


\subsection{Is there a (natural) optimal number of parcels that can be
identified from rsfMRI data}

% \begin{figure}
%     \centering
%     \includegraphics[width=0.5\linewidth]{}
%     \caption{Caption}
%     \label{fig:placeholder}
% \end{figure}


To address this question, we evaluated the performance of the Untamed
atlas in identifying regions of relatively homogeneous functional
activity, as measured by RSFC homogeneity (described in Section~\ref{sec:metric_rsfc_homo})
and task contrast variance (described in Section~\ref{sec:metric_alignment-with-task-contrasts}), relative to
random parcellations with an equal number of parcels on the HCP dataset.
Random parcellations were employed as a null model to assess whether
there was an optimal number of parcels for functional parcellation.

Specifically, we calculated the ratio of the metric scores obtained with
the Untamed atlas to those from random parcellations for both RSFC
homogeneity and task variance. Random parcellations were generated using
a region-growing process implemented in the MNE-Python package
\cite{Gramfort2013}. This process was repeated across a range of parcel
numbers, from 1 to 500 parcels per hemisphere.

For each parcel number, 50 random parcellation realizations were
generated with different initialization seeds. The mean weighted average
metric scores across these 50 random parcellations were computed and
used to calculate the performance ratio with respect to the Untamed
atlas. These ratios were then plotted as a function of the number of
parcels to investigate the relationship between parcel number and the
performance of the Untamed atlas relative to the null model.

\hypertarget{results}{%
\section{Results}\label{results}}

\subsection{Resting-state functional connectivity (RSFC)
homogeneity}
\label{sec:metric_rsfc_homo}


\begin{figure}
    \centering
    \includegraphics[width=\linewidth]{figs/Fig_2.jpg}
    \caption{Weighted average RSFC homogeneity on the (a) Yale (b) HCP dataset. Each bar plot depicts the subject-wise RSFC weighted average homogeneity, averaged across test subjects, for each baseline and the Untamed atlas, with matched parcel numbers for the left and right hemispheres. The error bars represent the standard error across all subjects. Effect sizes are shown in Table~\ref{tab:rsfc_homo_cohen_d}}
    \label{untamed:rst_homo}
\end{figure}

Fig.~\ref{untamed:rst_homo} (a) (Yale dataset) and Fig.~\ref{untamed:rst_homo} (b) (HCP dataset) present the weighted average RSFC homogeneity as bar charts. Corresponding effect
sizes, quantified by Cohen's d, are reported in Table~\ref{tab:rsfc_homo_cohen_d}. In Fig.~\ref{untamed:rst_homo} (a),
results from the Yale dataset demonstrates that the Untamed method
outperformed baseline methods across all tested parcel numbers with
statistical significance \((p < .001,\ uncorrected)\). Similarly, Fig.~\ref{untamed:rst_homo}(b) shows that on the HCP dataset, the Untamed method consistently
achieved higher weighted average homogeneity compared to baseline
methods, with most differences being statistically significant
\((p < .001)\), except for the 300-parcel case, where no statistically
significant differences were observed between Untamed-300 and
Schaefer-300 or Yan-300.

An exception is noted in the 400-parcel case on the HCP dataset, where
Schaefer-400 and Yan-400 achieve slightly higher homogeneity values than
the Untamed method. However, the differences are negligible, with no
variation up to the third decimal place and effect sizes of 0.0151 and
0.0085, respectively. Overall, these findings highlight the superior
performance of the Untamed method, which generally produces more
functionally homogeneous brain parcellations than other widely used
approaches. This superiority was consistently observed across different
datasets, parcel numbers, and parcellation schemes, emphasizing the
robustness and effectiveness of the Untamed atlases.

Interestingly, despite the large sample size of the HCP dataset, no
statistical significance is detected between Untamed-300 and either
Schaefer-300 or Yan-300. This suggests negligible differences between
these methods in the 300-parcel case. This observation is further
explored in the Discussion section.



\subsection{Alignment with task contrasts}
\label{sec:metric_alignment-with-task-contrasts}

\begin{figure}
    \centering
    \includegraphics[width=\linewidth]{figs/Fig_3.jpg}
    \caption{Weighted average task contrast variance evaluated on (a) MDTB (b) HCP test subjects. Each violin plot depicts the alignment with task contrast maps, computed for the baseline and the Untamed atlas, with matched parcel numbers for the left and right hemispheres.}
    \label{untamed:rst_contrast}
\end{figure}

The violin plots in Fig.~\ref{untamed:rst_contrast} (a, b) depict the distribution of task contrast variance (as defined in Eq. 4) for the Untamed method and
baseline atlases with matching parcel numbers. Fig. ~\ref{untamed:rst_contrast} (a) represents results from the MDTB task dataset, while Fig.~\ref{untamed:rst_contrast} (b) illustrates results from the HCP task dataset. Lower task contrast variances correspond to a smaller overall contrast variance per parcel, indicating better alignment of parcellation with functional task responses.

The Untamed method demonstrates statistically significant improvements
over baseline methods across almost all comparisons on the HCP dataset.
The only exception is Schaefer-300, which shows a lower task contrast
variance than its Untamed counterpart; however, this difference does not
reach statistical significance after applying Bonferroni correction for
multiple comparisons (raw\(\ p > .005\)). On the MDTB dataset, only
Schaefer-300 and Schaefer-400 show better alignment than their Untamed
counterparts with statistical significance. However, the differences are
indistinguishable up to the third decimal place. For the remaining
comparisons, Untamed either outperforms baseline methods with
statistical significance or shows comparable performance (e.g.,
Schaefer-100/200, Yan-300/400, where no significant differences are
observed).

Although Glasser-360 explicitly used HCP task fMRI data (in combination
with structural data) in constructing the atlas, the variances computed
on both the HCP and MDTB task data were significantly higher than those
of Untamed.

Fig.~\ref{untamed:fig:compare_contrast_vs_schaefer} (a, b) illustrates the overlay of Schaefer-100 and Untamed-100
parcels on the HCP group average "story" contrast map from the language
task. Regions near Broca's area and other auditory regions demonstrate
superior alignment of the Untamed parcels with contrast compared to
Schaefer's parcellation. Conversely, a counterexample is presented in
Figures~\ref{untamed:fig:compare_contrast_vs_schaefer} (c, d), where the "relational\_match" contrast reveals better
alignment of task-active regions with the Schaefer parcellation. The
weighted average variance of language\_story contrast is 4.3892 for
Untamed-100 and 5.4311 for Schaefer-100. Meanwhile, the weighted average variance of relational\_match contrast is 8.4673 for Untamed-100 and 8.0539 for Schaefer-100 (lower variance indicates better alignment).

Overall, the Untamed method consistently achieves lower weighted average
variance compared to baseline methods across most parcel numbers and
parcellation schemes. This result highlights the ability of the Untamed
method to produce functionally homogeneous brain regions with reduced
variability in functional task responses.

\subsection{Ablation study of input spatial maps: NASCAR versus ICA}

Fig.~\ref{untamed:homo_vs_ica} shows the weighted average RSFC homogeneity of the 1000 subjects
in the HCP dataset. The results indicates that despite the fact that
using the ICA maps from HCP has the advantage of same-dataset
hyperparameter fune-tining and remaining in the same surface space,
using input spatial maps from NASCAR still consistently outperformed ICA
across all four parcel numbers with statistical significance
\((p < .001)\).

\begin{figure}[htbp]
  \centering
  \includegraphics[width=\linewidth]{figs/Fig_5.jpg}
  \caption{Weighted average RSFC homogeneity on the HCP dataset. Each bar plot depicts the subject-wise RSFC weighted average homogeneity, averaged across test subjects, for each atlas generated from ICA maps and the Untamed atlas (generated from NASCAR maps). Parcel numbers matched for the left and right hemispheres. The error bars represent the standard error across all subjects. The ICA baselines, derived from and optimized through hyperparameter tuning on the same HCP dataset, benefit from this dataset-specific tuning. In contrast, the Untamed atlas was generated using the independent GSP dataset with hyperparameters tuned on GSP. The ICA results represent the maximum homogeneity achievable under optimal hyperparameter settings for the HCP dataset.}
  \label{untamed:homo_vs_ica}
\end{figure}

\subsection{Ablation study of graph embedding methods}

In Fig.~\ref{untamed:ablation_vs_spectral_S1}, we show parcellations obtained using graph node embedding
(NetMF) prior to clustering as described above with results obtained
using spectral clustering directly from the eigenvectors of the graph
Laplacian (GLC). Again, we show the bar plots of Untamed and GLC-based
atlases in RSFC weighted average homogeneity per HCP test. Among the 4
different numbers of parcels tested, NetMF-based Untamed outperforms the
GLC-based one in 3 cases with statistical significance \((p < .001)\). These results support the use of NetMF embedding in place of the more standard GLC approach.


\begin{figure}
    \centering
    \includegraphics[width=\linewidth]{figs/Fig_S1.jpg}
    \caption{Weighted average task contrast variance evaluated on (a) MDTB (b) HCP test subjects. Each violin plot depicts the alignment with task contrast maps, computed for the baseline and the Untamed atlas, with matched parcel numbers for the left and right hemispheres.}
    \label{untamed:ablation_vs_spectral_S1}
\end{figure}


\subsection{Ablation study of graph construction methods}
\label{untamed:sec:ablation_graph_construction}

\begin{figure}[htbp]
  \centering
  \includegraphics[width=\linewidth]{figs/Fig_4.jpg}
  \caption{Example HCP group average task activation $z$-score maps for Schaefer-100 and Untamed-100. First row: \textit{language\_story} contrast overlaid on boundaries (black) of (a) Schaefer-100 and (b) Untamed-100. Second row: \textit{relational\_match} contrast overlaid on (c) Schaefer-100 and (d) Untamed-100.}
  \label{untamed:fig:compare_contrast_vs_schaefer}
\end{figure}




\begin{figure}[htbp]
    \centering
    \includegraphics[width=\linewidth]{figs/Fig_S2.jpg}
    \caption{Weighted average task contrast variance evaluated on (a) MDTB (b) HCP test subjects. Each violin plot depicts the alignment with task contrast maps, computed for the baseline and the Untamed atlas, with matched parcel numbers for the left and right hemispheres.}
    \label{untamed:ablation_vs_graph_S2}
\end{figure}

We also explored the effect of graph construction as described in Section~\ref{untamed:sec:ablation_graph_construction} by comparing results using the NASCAR-based adjacency matrix with
that computed using the correlation of the RSFC matrix. All other
aspects of processing were identical. Fig.~\ref{untamed:ablation_vs_graph_S2} shows the bar plots of RSFC weighted average homogeneity of NASCAR-based and RSFC-based
methods. Evidently, the parcellations generated from the NASCAR-based
adjacency matrix substantially outperformed those generated from
Pearson-based adjacency in all cases \((p < .001)\). This demonstrated
an advantage of using the results of NASCAR tensor decomposition to
identify spatial networks over directly using the correlation of RSFC.


\subsection{Is there an optimal number of parcels?}

\begin{figure}
    \centering
    \includegraphics[width=6in]{figs/Fig_6.jpg}
    \caption{Ratios of three evaluation metrics comparing Untamed and random parcellations across 1 to 500 parcels per hemisphere: (a) Ratio of weighted average RSFC homogeneity: a comparison between Untamed parcellations and the mean values from 50 random parcellation trials (b) Ratio of weighted average task contrast variance: a comparison between Untamed parcellations and the mean values from 50 random parcellation trials.}
    \label{untamed:ratio_vs_random}
\end{figure}



\begin{figure}
    \centering
    \includegraphics[width=4.5in]{figs/Fig_7.jpg}
    \caption{Parcel-wise RSFC homogeneity scores (averaged across the rsfMRI data of 1000 HCP subjects) visualized on the parcel boundaries for (a) Untamed-300 and (b) Schaefer-300 (c) Comparison of parcel-wise homogeneity scores between the two atlases, with parcels matched using the Hungarian matching algorithm. The rank-ordered Dice coefficients between matched parcels are also shown.}
    \label{untamed:Fig_7}
\end{figure}



Fig.~\ref{untamed:ratio_vs_random} illustrates the ratio of RSFC homogeneity and task contrast variance
between the Untamed atlas and random parcellations for the HCP task
dataset. For RSFC homogeneity, a larger ratio indicates comparatively
higher homogeneity for the Untamed atlas relative to its random
parcellation counterpart. Conversely, for task contrast variance, a
smaller ratio indicates better performance by the Untamed atlas.

Both curves exhibit a similar trend: the advantage of the Untamed atlas
over random parcellations increases initially, peaks, and then
diminishes as the number of parcels continues to grow. The optimal
number of parcels varies between the two modalities. Fig.~\ref{untamed:ratio_vs_random} (a) indicates that the greatest relative advantage of the Untamed atlas over random parcellations occurs at fewer than 50 parcels per hemisphere for
the homogeneity metrics as evidenced by the peak. The contrast variance ratio in Fig.~\ref{untamed:ratio_vs_random} (b) remains relatively flat and close to its minimum from approximately 50-150 parcels with the advantage relative to a
random parcellation diminishing approximately monotonically above 200
parcels.


\section{Discussion}\label{discussion}

This paper introduced \emph{Untamed}, a novel cortical parcellation
scheme developed from population resting-state fMRI data. This scheme constructed spatially disjoint parcels by leveraging the overlapping and
correlated brain networks identified by NASCAR tensor decomposition \cite{Li2021, Li2023}. We compared Untamed to an extensive list of popular
atlases and parcellation methods, as listed in Table~\ref{tab:atlas_baselines}, the visualization comparison of Untamed and Schaefer with varying parcel numbers are in Fig.~\ref{fig:untamed:parcel_vis_vs_schaefer_S5}. As noted in Section~\ref{sec:metric_rsfc_homo}, the weighted average RSFC homogeneity among the HCP subjects revealed no statistically significant differences between Untamed-300 and Schaefer-300, nor between Untamed-300 and Yan-300. The large sample size of 1000 test subjects indicated that the differences in homogeneity between Untamed and these two atlases at this particular parcel resolution were minimal. Nonetheless, a more detailed examination of the per-parcel RSFC homogeneity revealed nuanced differences. We conducted an in-depth comparison between Untamed-300 and Schaefer-300.
As illustrated in Fig.~\ref{untamed:rsfc_vs_schaefer200} (a) and (b), the overall spatial distribution of per-parcel RSFC homogeneity exhibited a broadly similar pattern across the cortex for both parcellations. Specifically, in both cases, parcels in the parietal and occipital lobes exhibit higher RSFC
homogeneity compared to those in the frontal lobe. This may be attributed to the more functionally specialized regions in the parietal
and occipital areas, which contribute to stronger intra-parcel connectivity, whereas the frontal lobe encompasses more heterogeneous and integrative functions. Despite these general similarities, visible differences in per-parcel RSFC homogeneity were observed across several cortical areas. 





\begin{figure}
    \centering
    \includegraphics[width=6.5in]{figs/Fig_9.jpg} 
    \caption{RSFC based on (a) Untamed-200 and (b) Schaefer-200 (averaged across subjects) on Yale rsfMRI test subjects after global signal regression. Both clustered to 7 networks as in (Yeo et al., 2011) using spectral clustering. (c) and (d) shows all parcels of Untamed-200 and Shaefer-200 assigned network colors.}
    \label{untamed:rsfc_vs_schaefer200}
\end{figure}


\begin{figure}
    \centering
     \includegraphics[width=\linewidth, height=0.9\textheight, keepaspectratio]{figs/Fig_S3.jpg} 
    \caption{Parcel-wise RSFC homogeneity scores (averaged across the rsfMRI data of 1000 HCP subjects) visualized on the parcel boundaries for (a) Untamed-100 and (b) Schaefer-100. (c) Comparison of parcel-wise homogeneity scores between the two atlases, with parcels matched using the Hungarian matching algorithm. The rank-ordered Dice coefficients between matched parcels are also shown. }
    \label{parcel_wise_homo_vs_schaefer_S3}
\end{figure}

Fig. 7 (c) quantifies these differences, highlighting variations in RSFC homogeneity values between parcels for the two parcellations.
While the weighted average RSFC homogeneity reported in section 3.1
provides a global perspective on performance across the entire cortex,
per-parcel RSFC homogeneity scores offer a finer-grained view. Notably,
the per-parcel RSFC homogeneity scores between Untamed and Schaefer
remained significantly more consistent compared to those between Untamed
and a random parcellation (as described in Section 2.10). For easier
visual comparison, we provide per-parcel homogeneity plots similar in
style to Fig. 7, illustrating Untamed-100 vs. Schaefer-100 and
Untamed-100 vs. Random-100 in Fig.~\ref{parcel_wise_homo_vs_schaefer_S3} and ~\ref{parcel_wise_homo_vs_random_S4}. The random parcellation
showed substantially less alignment, both in terms of parcel overlap and parcel-wise RSFC homogeneity.


\begin{figure}
    \centering
     \includegraphics[width=\linewidth, height=0.9\textheight, keepaspectratio]{figs/Fig_S4.jpg} 
    \caption{Parcel-wise RSFC homogeneity scores (averaged across the rsfMRI data of 1000 HCP subjects) visualized on the parcel boundaries for (a) Untamed-100 and (b) Random-100. (c) Comparison of parcel-wise homogeneity scores between the two atlases, with parcels matched using the Hungarian matching algorithm. The rank-ordered Dice coefficients between matched parcels are also shown. }
    \label{parcel_wise_homo_vs_random_S4}
\end{figure}


We further quantify the similarity of parcel assignments using the
Adjusted Rank Index (ARI) across groups within Untamed and between
Untamed and other atlases (Fig.~\ref{untamed:ARI}). ARI measures the agreement between
two clustering solutions by considering all pairs of samples and
counting the number of pairs assigned to the same or different clusters.
For within-Untamed comparison, we randomly split the full set of GSP subjects into two equal subgroups (Group 1 and 2, each with 714
subjects) and generated separate atlases using the same pipeline
described in the Methods section. We then assessed reproducibility by
computing ARI between the atlases from Group 1 and Group 2 (Fig.~\ref{untamed:ARI} (1)),
and between each group-specific atlas and the one generated using all
subjects (Fig.~\ref{untamed:ARI} (2)). Comparisons within Untamed show the highest
agreement, both across subject groups and varying sample sizes, while
comparisons with existing atlases (Schaefer, Yan) yield moderate
alignment. Agreement with Random parcellations is consistently the
lowest across all cases. Notably, the Yan atlas tends to exhibit
slightly higher alignment with the Schaefer atlas, likely due to their
use of similar Markov Random Field-based algorithms and shared dataset
in generating parcels.

\begin{figure}
    \centering
    \includegraphics[width=5in]{figs/Fig_8.jpg}
    \caption{Adjusted Rand Index (ARI) across parcel numbers. The full set of GSP subjects was randomly split into two equal subgroups (Group 1 and Group 2, each with 714 subjects). The figure presents ARI scores between: (1) Untamed Group 1 v.s. Group 2, (2) Untamed subgroups v.s. All, (3) Schaefer v.s. Yan, (4) Untamed v.s. Random, (5) Untamed v.s. Schaefer, and (6) Untamed v.s. Yan.}
    \label{untamed:ARI}
\end{figure}

To investigate the relationship between the spatially contiguous parcels
and distributed networks, we first computed the RSFC between parcels for
the Untamed-200 and Schaefer-200 atlases and then applied an automated
spectral clustering \cite{Ng2001} method to identify 7 networks
consisting of groups of parcels that exhibit the highest similarity in
their connectivity patterns. We used the Yale rsfMRI dataset and
performed a global signal regression before computing the Pearson
correlation between parcels on signals formed by averaging the rsfMRI
signal across each parcel. The ordered RSFC matrices and the
corresponding set of 7 networks (color-coded) for each atlas are shown
in Fig.~\ref{untamed:rsfc_vs_schaefer200}. The spatial boundaries of the networks assigned to parcels
shared a similar trend between the two based on the RSFC matrices (Fig.~\ref{untamed:rsfc_vs_schaefer200} (a)(b)). By calculating the weighted average RSFC homogeneity using
the 200 parcels, as described in Section 2.6.1, but applied within each
of the 7 networks rather than within parcels for vertices, we found that
the 7 networks derived from Untamed-200 achieved a value of
\(0.2077 \pm 0.0143\), slightly higher than the corresponding
homogeneity value for Schaefer-200, which is \(0.2042 \pm 0.0180\) (mean
\(\pm \ \)standard deviation) across the Yale rsfMRI subjects
(\(p = 0.0696\) with paired sample t-test). Additionally, when
evaluating the alignment of network assignments with Yeo's 7 networks \cite{Yeo2011} by projecting the network assignments back to each
vertex, the Adjusted Rand Index (ARI) between Untamed-200 and Yeo-7 was
0.3742, slightly higher than that between Schaefer-200 and Yeo-7, which
was 0.3605. This indicates that the network assignments from Untamed-200
are slightly more aligned with Yeo's 7 networks than Schaefer-200.
However, this general trend varies across specific networks. For
instance, while the visual and dorsal part of the somatomotor networks
from Schaefer-200 are more similar to the original Yeo-7 parcellation,
the frontoparietal network and ventral part of the somatomotor network
from Untamed-200 are more similar to the original Yeo-7. These findings
suggested nuanced differences in network delineation between the two
methods, with strengths and limitations varying across specific
networks.

Revisiting the NASCAR networks that inform the Untamed parcels, Fig.~\ref{untamed:nascar_overlay}
provides a visual depiction of both the 100 and 300 parcel versions of
Untamed, superimposed on two default mode sub-networks, the visual
network and the somatomotor network derived from NASCAR. The parcel
boundaries of Untamed generally align closely with the transition
between activated and deactivated regions in the NASCAR networks,
validating that the spatial maps indeed guide the parcellations.
Additionally, Fig.~\ref{untamed:nascar_overlay}  indicates that a cortical vertex can belong to
multiple functional networks, illustrating the overlapping nature of the
NASCAR networks. We further explored the relationship of network
participation among different vertices. The spatial correlation of the NASCAR spatial maps

\begin{equation}
    \mathbf{A} = corr(\mathbf{X}) \in \mathbb{R}^{\mathbf{|V|} \times \mathbf{|V|}}
\end{equation}

computed as the initial step in graph construction described in Section
2.4, quantifies the similarity of network participation among vertices.
We retained all the non-negative values in this map and computed the
degree, which is the sum of all similarity values between each vertex
with all other vertices. This was then normalized by subtracting the
minimum degree value and dividing by the range between the maximum and
minimum degree values, thus scaling the values to fall within {[}0,
1{]}. The resulting degree distribution across the entire cortex is
illustrated in Fig. 11. It indicates that (1) there is variability in
the network participation levels among vertices (2) the degree
distribution across the cortex shows high values predominantly in the
prefrontal, posteror cingulate, lateral temporal, and lateral parietal
cortex. These regions correspond to the functional hubs identified in previous literature \cite{Buckner2009, VanDenHeuvel2013}, underscoring the accuracy of the overlapping and correlated
NASCAR spatial maps in depicting network participation.

\begin{figure}
    \centering
    \includegraphics[width=6.5in]{figs/Fig_10.jpg}
    \caption{(a)(c) and (b)(d) depict two NASCAR default mode sub-networks with (a)(c) Untamed-100 (b)(d) Untamed-300 parcel boundaries; (e) and (f) display a NASCAR visual network with (c) Untamed-100 and (d) Untamed-300; (g) and (h) display NASCAR somatomotor network with (g) Untamed-100 and (h) Untamed-300.}
    \label{untamed:nascar_overlay}
\end{figure}


In this chapter, the Untamed atlases, generated via a fully automated
pipeline, demonstrated competitive performance in terms of RSFC
homogeneity and task contrast alignment. Despite the wide array of
existing brain atlases, Untamed proves its value by effectively
segregating the cortex into distinct regions while maximizing the
similarity of within-parcel connectivity and task activation patterns.
While atlases are commonly employed for reducing data dimensionality,
the selection process often appears somewhat arbitrary and relies on
assumptions \cite{Bryce2021}. Though atlases like Schaefer may
occasionally show superior alignment with specific task contrasts (e.g.,
relational\_match in Fig.~\ref{untamed:rst_contrast} or exhibit higher RSFC homogeneity at certain parcel resolutions, they can be a more suitable choice for
analyzing fMRI data when a context-specific application is required \cite{Moghimi2022}. However, Untamed generally offers excellent performance in RSFC homogeneity and task contrast alignment for various applications. Its fully automated nature facilitates easy application across different datasets, enhancing its utility in neuroscientific research.

\begin{figure}
    \centering
    \includegraphics[width=5in]{figs/Fig_11.jpg}
    \caption{Degree distribution across the cortex, calculated from the spatial correlation of the NASCAR spatial maps and normalized to range within 0 and 1.}
    \label{untamed:deg_distrib}
\end{figure}


\begin{figure}
    \centering
    \includegraphics[width=6.5in]{figs/Fig_S6.jpg}
    \caption{First row: Spatial correlation maps derived from pairwise Pearson correlations of the 50 spatial maps from NASCAR (left) and ICA (right). The ICA spatial maps were obtained from the HCP website (HCP 1200 project, computed from 1003 subjects). Second row: Histograms display the distribution of the upper triangular elements in each spatial correlation matrix. The ICA values are tightly clustered around zero, reflecting numerical noise. In contrast, the NASCAR values exhibit a broader distribution, indicating a lower degree of constraint imposed by the algorithm.}
    \label{untamed:spatial_corr_S6}
\end{figure}


Further, we leveraged our automated pipeline to explore several key
questions, including the impact of relaxing biologically implausible
constraints. We evaluated the efficacy of using functional network
spatial maps from NASCAR and ICA as inputs for informing parcellations.
In our ablation studies, we observed a clear advantage of using spatial
maps derived from NASCAR compared to those generated by ICA. Unlike ICA
and other data-driven methods, NASCAR spatial maps are unconstrained,
allowing for spatial correlations. Figure~\ref{untamed:spatial_corr_S6} illustrates the differences in spatial correlations between NASCAR and ICA maps. Our RSFC homogeneity comparisons (Fig.~\ref{untamed:homo_vs_ica}) revealed that NASCAR's removal of
spatial constraints resulted in consistently higher RSFC homogeneity
compared to ICA-derived maps. This demonstrated that the time series
within each parcel informed by networks removing constraints such as
statistical independence are more homogeneous. Additionally, we explored
another critical neuroscience question concerning the optimal number of
brain parcels \cite{Schaefer2018,Yan2023}. Our experiments revealed that the maximum benefits of using Untamed, in terms of RSFC homogeneity and task contrast alignment, are achieved with fewer than 200 parcels per hemisphere, as shown in Fig. 6. Beyond this number, the
advantages of using a principled approach over random seed-based region-growing slowly diminish. This suggests that increasing parcellation density will, at some point, offer diminishing returns in
terms of effectively defining meaningful subdivisions.

Methodologically, Untamed consists of three steps once the rsfMRI are
processed and spatially aligned: (i) BrainSync synchronization of rsfMRI
data, (ii) NASCAR-based tensor-decomposition, and (iii) NetMF-based
graph embedding and \(k\)-means clustering. To further enhance the
parcellations obtained using the framework described here, alternative
methods for factorization of the NetMF matrix (Eq. 2) could be explored.
Our current methodology follows the practice \cite{Qiu2018} that
utilizes the left singular vectors and singular values of the NetMF
matrix to construct low-dimensional embeddings. However, other
factorization techniques, such as stochastic matrix factorization, which
can incorporate weighting for different vertices, may potentially yield
improved results \cite{Levy2014}.

\begin{figure}
    \centering
    \includegraphics[width=6.5in]{figs/Fig_S5.jpg}
    \caption{Parcellation comparison using Schaefer’s method: (a)(c)(e)(g): Schaefer-100/200/300/400 and Untamed: (b)(d)(f)(h): Untamed-100/200/300/400. From top to bottom we show 100, 200, 300, and 400 parcels. At each resolution, Hungarian matching was performed to match between the two parcellations to find maximum correspondence in terms of vertex-wise label assignment.}
    \label{fig:untamed:parcel_vis_vs_schaefer_S5}
\end{figure}



\hypertarget{data-and-code-availability}{%
\section{\texorpdfstring{Data and code availability
}{Data and code availability }}\label{data-and-code-availability}}

The data used in this study are publicly available from the Genomics Superstruct Project (GSP) (\url{https://www.neuroinfo.org/gsp}) and the
Human Connectome Project, Young Adult Study (\url{https://www.humanconnectome.org/study/hcp-young-adult}). The MDTB dataset task contrast maps are
publicly available at \url{https://github.com/DiedrichsenLab/DCBC}. The
parcellations generated in this study and the associated code are
available at the \url{https://untamed-atlas.github.io}.



% \begin{center}
% \begin{table*}[!h]%
% \caption{This is sample table caption.\label{tab1}}
% \begin{tabular*}{\textwidth}{@{\extracolsep\fill}lllll@{}}
% \toprule
% &\multicolumn{2}{@{}l}{\textbf{Spanned heading$^{\tnote{\bf a}}$}} & \multicolumn{2}{@{}l}{\textbf{Spanned heading$^{\tnote{\bf b}}$}} \\\cmidrule{2-3}\cmidrule{4-5}
% \textbf{Col1 head} & \textbf{Col2 head}  & \textbf{Col3 head}  & {\textbf{Col4 head}}  & \textbf{Col5 head}   \\
% \midrule
% col1 text & col2 text  & col3 text  & 12.34  & col5 text\tnote{1}   \\
% col1 text & col2 text  & col3 text  & \hphantom{0}1.62  & col5 text\tnote{2}   \\
% col1 text & col2 text  & col3 text  & 51.809  & col5 text   \\
% \bottomrule
% \end{tabular*}
% \begin{tablenotes}%%[341pt]
% \item[$^{\rm a}$] Example for a first table footnote.
% \item[$^{\rm b}$] Example for a second table footnote.
% \item {\it Source}: Example for table source text.
% \end{tablenotes}
% \end{table*}
% \end{center}

\noindent\textbf{un-numbered list items sample:}


\begin{sidewaystable}%[h]
\def\d{\hphantom{0}}
\caption{Sideways table caption. For decimal alignment refer column 4 to 9 in tabular* preamble.\label{tab3}}%
\begin{tabular*}{\textheight}{@{\extracolsep\fill}lllllllll@{\extracolsep\fill}}%
\toprule
  & \textbf{Col2 head} & \textbf{Col3 head} & \multicolumn{1}{l}{\textbf{10}} &\multicolumn{1}{l}{\textbf{20}} &\multicolumn{1}{l}{\textbf{30}} &\multicolumn{1}{l}{\textbf{10}} &\multicolumn{1}{l}{\textbf{20}} &\multicolumn{1}{l}{\textbf{30}} \\
\midrule
  &col2 text &col3 text &\d0.7568&\d1.0530&\d1.2642&\d0.9919&\d1.3541&\d1.6108 \\
  &          &col2 text &12.5701 &19.6603&25.6809&18.0689&28.4865&37.3011 \\
3 &col2 text & col3 text &\d0.7426&\d1.0393&\d1.2507&\d0.9095&\d1.2524&\d1.4958 \\
  &          &col3 text &12.8008&19.9620&26.0324&16.6347&26.0843&34.0765 \\
  & col2 text& col3 text &\d0.7285&\d1.0257&\d1.2374&\d0.8195&\d1.1407&\d1.3694* \\
  &          & col3 text &13.0360&20.2690&26.3895&15.0812&23.4932&30.6060\tnote{$\dagger$}  \\
\bottomrule
\end{tabular*}
\begin{tablenotes}%%[\textheight]
\item[*] First sideways table footnote. Sideways table footnote. Sideways table footnote. Sideways table footnote.
\item[$^\dagger$] Second sideways table footnote. Sideways table footnote. Sideways table footnote. Sideways table footnote.
\end{tablenotes}
\end{sidewaystable}


\appendix

\bmsubsection{Subsection title of first appendix\label{app1.1a}}



\bmsubsubsection{Subsection title of first appendix\label{app1.1.1a}}

\noindent\textbf{Unnumbered figure}


\begin{center}
\includegraphics[width=7pc,height=8pc,draft]{empty}
\end{center}


\bmsection{Section title of second appendix\label{app2}}%
\vspace*{12pt}
Fusce mauris. Vestibulum luctus nibh at lectus. Sed bibendum, nulla a faucibus semper, leo velit ultricies tellus, ac
venenatis arcu wisi vel nisl. 

%== Figure 4 ==
%% Example for figure inside appendix
\begin{figure}[b]
\centerline{\includegraphics[height=10pc,width=78mm,draft]{empty}}
\caption{This is an example for appendix figure.\label{fig5}}
\end{figure}

\bmsubsection{Subsection title of second appendix\label{app2.1a}}


\begin{table*}[t]%
\centering
\caption{This is an example of Appendix table showing food requirements of army, navy and airforce.\label{tab4}}%
\begin{tabular*}{\textwidth}{@{\extracolsep\fill}llllll@{\extracolsep\fill}}%
\toprule
\textbf{Col1 head} & \textbf{Col2 head} & \textbf{Col3 head} & \textbf{Col4 head} & \textbf{Col5 head} & \textbf{Col6 head} \\
\midrule
col1 text & col2 text & col3 text & col4 text & col5 text & col6 text\\
col1 text & col2 text & col3 text & col4 text & col5 text & col6 text\\
col1 text & col2 text & col3 text& col4 text & col5 text & col6 text\\
\bottomrule
\end{tabular*}
\end{table*}


Example for an equation inside appendix
\begin{equation}
{\mathcal{L}} = i \bar{\psi} \gamma^\mu D_\mu \psi - \frac{1}{4} F_{\mu\nu}^a F^{a\mu\nu} - m \bar{\psi} \psi\label{eq25}
\end{equation}

% \begin{center}
% \begin{tabular*}{250pt}{@{\extracolsep\fill}lcc@{\extracolsep\fill}}%
% \toprule
% \textbf{Col1 head} & \textbf{Col2 head} & \textbf{Col3 head} \\
% \midrule
% col1 text & col2 text & col3 text \\
% col1 text & col2 text & col3 text \\
% col1 text & col2 text & col3 text \\
% \bottomrule
% \end{tabular*}
% \end{center}

\bibliography{wileyNJD-AMA}

% \nocite{*}% Show all bib entries - both cited and uncited; comment this line to view only cited bib entries;

\end{document}
